\chapter*{Введение}
\addcontentsline{toc}{chapter}{Введение}

Цель практики -- изучение и практическое применение инструментов для автоматизации и поддержки процессов непрерывной интеграции и непрерывного развертывания программного обеспечения в процессе разработки.

В рамках реализации проекта должны быть решены следующие задачи:
\begin{itemize}
	\item изучить документацию Gitlab CI/CD, Docker, Qt, Qt Test, Qt Widgets, FFmpeg;
    \item cоздать программу, принимающую данные о сцене и создающую изображение с помощью алгоритма, использующего Z-буфер;
    \item реализовать управление программой из командной строки;
    \item cоздать сценарий gitlab-ci.yml автоматизации сборки, тестирования и получения
    данных будущего исследования; 
    \item выбрать готовые образы docker для задач из сценария;
    \item cоздать три сцены для демонстрации работоспособности всего конвейера;
    \item создать модульный тест для демонстрации работоспособности системы;
    \item создать один сценарий исследования зависимости времени выполнения программы от количества полигонов на сцене.
\end{itemize}

