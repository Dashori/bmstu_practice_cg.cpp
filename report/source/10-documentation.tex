\chapter{Знакомство с документацией}
В данном разделе представлены ссылки и краткое содержание документации, которая была изучена в рамках практики.
\section{Gitlab CI/CD}
Была изучена официальная документация Gitlab CI/CD \cite{bib1}.

Gitlab CI/CD -- это инструмент для разработки программного обеспечения с использованием непрерывных методологий:
\begin{itemize}
	\item непрерывная интеграция (Continuous Integration);
	\item непрерывная доставка (Continuous Delivery);
	\item непрерывное развертывание (Continuous Deployment).
\end{itemize}

Чтобы использовать GitLab CI/CD необходим код приложения, размещенный в репозитории Git, и файл .gitlab-ci.yml, находящийся в корне репозитория и содержащий конфигурацию CI/CD. 

Используемые концепции и их описание:
\begin{itemize}
	\item Файл .gitlab-ci.yml
	
		Используя данный файл GitLab Runner запускает конвейеры и сценарии, определенные в заданиях.
		
		Были изучены такие ключевые слова, как:
		\begin{itemize}
			\item image(docker образ);
			\item stage(стадии конвейера);
			\item before\_script(команды, выполняемые перед основным скриптом);
			\item script(команды основного скрипта);
			\item artifacts(данные, полученные в результате заданий);
			\item needs(зависимости между заданиями).
		\end{itemize}
		
		\begin{lstlisting}[caption={Пример файла .gitlab-ci.yml}]
graph:
	stage: result_graph
	image: python
	before_script:
		- pip install matplotlib
	script:
		- python3 result_graph.py
	artifacts:
		paths:
		- chicago_develop/result.png
		expire_in: 1 day
	needs:
	- research
		\end{lstlisting}		
	\item Конвейеры(pipelines)
	
	Конвейеры или сборочные линии содержат стадии(stages), которые определяют задания и когда их следует запускать. Различают стадии сборки, тестирования, организации и выпуска.
	
	\item Задания(jobs)
	
	Настройка конвейера начинается с заданий. Задания являются фундаментальным элементом файла gitlab-ci.yml.
	
	Задания:
	\begin{itemize}
	    \item определены с ограничениями, указывающими, при каких условиях они должны выполняться;
		\item элементы верхнего уровня с произвольным именем должны содержать секцию script;
		\item не ограничены в количестве.
	\end{itemize}

	\begin{lstlisting}[caption={Пример задания}]
debug-build:
stage: build_debug
	script: make
	\end{lstlisting}	
	""\newline
	\item Артефакты задания(job artifacts)
	
	Артфекаты задания -- архив файлов и каталогов, которые могут выпускать задания. Их можно загружать с помощью пользовательского интерфейса GitLab или API.
	
	\begin{lstlisting}[caption={Пример создания артефактов задания}]
generate-image:
script: ./app.exe -image 
artifacts:
	paths:
	- /pictures/image.png
	expire_in: 7 days
	\end{lstlisting}

	В этом примере задание с именем generate-image запускает приложение app.exe, которое генерирует изображение. Ключевое слово paths определяет, какие файлы следует добавлять
	в артефакты задания. Все пути к файлам и каталогам относятся к репозиторию, в котором было создано задание. Ключевое слово expire\_in определяет, как долго GitLab хранит артефакты задания.
	
\end{itemize}
\section{Docker}
Docker -- программное обеспечение для автоматизации развёртывания и управления приложениями в средах с  поддержкой контейнеризации.

Была изучена официальная документация Docker \cite{bib2}:
\begin{itemize}
	\item Создание Dockerfile

\begin{lstlisting}[caption={Dockerfile}]
# syntax=docker/dockerfile:1
FROM python:3.7 - alpine
WORKDIR /code
ENV FLASK_APP=app.py
ENV FLASK_RUN_HOST=0.0.0.0
RUN apk add --no-cache gcc musl-dev linux-headers
COPY requirements.txt requirements.txt
RUN pipinstall -r requirements.txt
EXPOSE 5000
COPY . .
CMD ["flask", "run"]
\end{lstlisting}

В этом примере докер:
\begin{itemize}
	\item cоздаёт образ, начиная с образа Python 3.7;
	\item устанавливает рабочий каталог/code;
	\item задаёт переменные среды, используемые командой flask;
	\item устанавливает gcc и другие зависимости;
	\item копирует requirements.txt и устанавливает зависимости Python;
	\item добавляет метаданные к образу, чтобы описать, что контейнер прослушивает порт 5000;
	\item копирует текущий каталог . в проекте в рабочий каталог . в образе;
	\item устанавливает для контейнера команду по умолчанию flask run.
\end{itemize}
	
	\item Определение служб в Compose file
	
	\begin{lstlisting}[caption={Compose file}]
version: "3.9"
services:
	web:
		build: .
		ports:
			-"8000:5000"
	redis:
		image: "redis:alpine"	
	\end{lstlisting}
Compose file определяет две службы: web и redits. Веб-служба использует образ, созданный из файла Dockerfile в текущем каталоге. Затем он привязывает контейнер и хост-машину к открытому порту 8000. В этом примере службы используется порт по умолчанию для веб-сервера Flask 5000. Служба redis использует общедоступный образ Redis, извлеченный из реестра Docker Hub.
	\item Сборка и запуск приложения с помощью Compose
		
	Из каталога проекта было запущено приложение с помощью docker compose up.
	\item Docker Hub
	
	Были изучены создание, удаление, клонирование репозиториев в Docker Hub \cite{bib3}.
\end{itemize}


\section{QT}
Были изучены интерфейсы основных классов библиотеки Qt \cite{bib4}, необходимые для работы с компьютерной графикой:
\begin{itemize}
	\item QImage
	
	Класс QImage обеспечивает аппаратно-независимое представление изображения. QImage разработан и оптимизирован для ввода-вывода, а также для прямого доступа к пикселям и манипулирования ими. Также позволяет сохранять изображения в формате PNG.
	
	\item QPixmap
	
	Класс QPixmap разработан и оптимизирован для отображения изображения на экране, которое можно использовать в качестве устройства рисования. 
	
	\item QVector3D	
	
	Класс QVector3D представляет вектор или вершину в трехмерном пространстве.
	 
	Помимо конструкторов создания были изучены и использованы методы:
	\begin{itemize}
		\item dotProduct(QVector3D v1, QVector3D v2) -- скалярное произведение векторов v1 и v2;
		\item normalize() -- нормирование вектора;
		\item toVector4D() -- четырехмерная форма трехмерного вектора с нулевой координатой w;
		\item setX(float x) -- установка координаты x;
		\item setY(float y) -- установка координаты y;
	 	\item setZ(float z) -- установка координаты z.
	\end{itemize}
	
	\item QVector4D
	
	Класс QVector4D представляет вектор или вершину в четырехмерном пространстве. Был использован для матриц преобразований в трехмерном пространстве.
	
	\item QColor
	
	Класс QColor предоставляет цвета на основе значений RGB, HSV или CMYK. QColor был использован в программе для создания и изменения цвета на основе значений RGB. 
\end{itemize}
\section{QT Test}

Qt Test \cite{bib5} -- это фреймворк для модульного тестирования приложений и библиотек на основе Qt. Qt Test предоставляет расширения для тестирования графических пользовательских интерфейсов.

Были изучен и использован макрос QCOMPARE(actual, expected). Он сравнивает фактическое значение с ожидаемым значением с помощью оператора равенства. Если фактическое и ожидаемое совпадают -- выполнение продолжается. Иначе ошибка записывается в журнал тестирования, и тестовая функция возвращается без попыток каких-либо последующих проверок.

\section{QT Widgets}
Qt Widgets \cite{bib6} предоставляет набор элементов пользовательского интерфейса для создания классических пользовательских интерфейсов. 

Были изучены основные классы модуля:
\begin{itemize}
	
	\item QApplication 
	
	Класс QApplication управляет потоком управления и основными настройками приложения с графическим интерфейсом. Для любого приложения с графическим интерфейсом, использующего Qt, существует ровно один объект QApplication, независимо от того, имеет ли приложение 0, 1, 2 или более окон в любой момент времени. 
	
	\item QMainWindow
	
	Класс QMainWindow предоставляет главное окно приложения. Главное окно обеспечивает основу для создания пользовательского интерфейса приложения.
	
	\item QGraphicsScene
	
	Класс QGraphicsScene предоставляет поверхность для управления большим количеством 2D-графических элементов. Он используется вместе с QGraphicsView для визуализации графических элементов, таких как линии, прямоугольники, текст или даже пользовательские элементы, на 2D-поверхности.
	
	\item QGraphicsView
	
	Класс QGraphicsView предоставляет виджет для отображения содержимого. QGraphicsView визуализирует содержимое QGraphicsScene в прокручиваемом окне просмотра. 
		
	\item QPushButton
	
	Виджет, представляющий обычную кнопку. Если произошло событие clicked, управление передаётся обработчику этого события.
	
\end{itemize}


\section{FFmpeg}
Была изучена официальная документация утилиты FFmpeg \cite{bib7}. Она позволяет записывать, конвертировать и передавать цифровые аудио- и видеозаписи в различных форматах.
	\begin{lstlisting}[caption={Пример создания MP4 видеофильма из PNG изображений}]
ffmpeg -r 5 -f image2 -s 1920x1080 -i film%d.png -crf 1 -pix_fmt yuv420p film.mp4
	\end{lstlisting}
	\begin{itemize}
	\item -r – установка частоты кадров в секунду;
	\item -f – установка имени файла форматом film\%02d.png;
\	\item -s – установка размера кадра;
	\item -i – ввод имени входного файла;
	\item -crf – фактор постоянного оценивания (Constant Rate Factor);
	\item -pix\_fmt – установка формата пикселей.
	\end{itemize}